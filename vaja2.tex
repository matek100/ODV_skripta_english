\chapter{Priprava na 2. laboratorijske vaje}

\section{Zapis preklopnih funkcij}

%Za $n$ vhodnih spremenljivk obstaja $2^{2^n}$ preklopnih funkcij. Preklopno funkcijo lahko podamo na štiri načine:
%\begin{enumerate}
%\item tabelarični zapis (pravilnostna tabela),
%\item analitični zapis (z logičnim izrazom),
%\item matrični zapis (Veitchevi diagrami),
%\item logična shema (grafično).
%\end{enumerate}

Analitični zapis podajamo z logičnim izrazom oziroma logično enačbo. Pogosto se uporabljajo tri posebne oblike analitičnega zapisa, in sicer normalna, popolna in minimalna. Preklopna funkcija je podana v normalni obliki, če je sestavljena iz največ dveh nivojev logičnih operatorjev. Oblika je popolna, če na prvem nivoju logičnih operatorjev v vseh izrazih nastopajo vse vhodne spremenljivke. Minimalna oblika je najkrajša možna oblika zapisa preklopne funkcije.

\section{Analitični zapis in popolne normalne oblike zapisa preklopnih funkcij}

Pogledali si bomo popolno disjunktivno normalno obliko (PDNO) in popolno konjunktivno normalno obliko (PKNO) zapisa preklopne funkcije.

\subsection{Popolna disjunktivna normalna oblika (PDNO)}
\begin{itemize}
\item Disjunktivna: operator 2. nivoja je disjunkcija ($\vee$).
\item $f(x_1,x_2,...,x_n)=\vee_{i=0}^{2^{n}-1}m_i f(\vec{w}_i)$
\item $f(\vec{w}_i)$ ... vrednost funkcije pri $i$-tem vhodnem vektorju (vrstici)
\item minterm je konjunktiven izraz - vse vhodne spremenljivke so povezane preko konjunkcije
\item $m_i$ ... minterm $i$; $m_i = x_1^{w_{1,i}} \cdot x_2^{w_{2,i}} \cdot ... \cdot x_n^{w_{n,i}}; i=0,1,2,...,2^n-1$
\item $x^w = \left\{\begin{array}{clcr}
 x, w=1 \\ \ol x, w=0 \end{array} \right.$
\item $w_{j,i}$...$j$-ti bit binarnega zapisa števila $i$
\item PDNO lahko zapišemo v krajši obliki kot $f(x_1,x_2,...,x_n)=\vee^n(i_1,i_2,...,i_k)$, kjer $i_1,i_2,...,i_k$ določajo indekse mintermov, ki nastopajo v PDNO.
\end{itemize}

Recept: pri določanju PDNO disjunktivno vežemo tiste minterme, pri katerih je funkcijska vrednost 1. Pri tem se $i$-ti minterm nanaša na funkcijsko vrednost $f(\vec{w}_i)$ (glej tabelo \ref{tab:mintermi}).

\begin{table}[ht]
\centering
\begin{tabular}{ccc|c|c}
$x_1$ & $x_2$ & $x_3$ & f($x_1$,$x_2$,$x_3$) & minterm\\
\hline
0 & 0 & 0 & $f(\vec{w}_0)$ & $m_0$ \\
0 & 0 & 1 & $f(\vec{w}_1)$ & $m_1$ \\
0 & 1 & 0 & $f(\vec{w}_2)$ & $m_2$ \\
0 & 1 & 1 & $f(\vec{w}_3)$ & $m_3$ \\
1 & 0 & 0 & $f(\vec{w}_4)$ & $m_4$ \\
1 & 0 & 1 & $f(\vec{w}_5)$ & $m_5$ \\
1 & 1 & 0 & $f(\vec{w}_6)$ & $m_6$ \\
1 & 1 & 1 & $f(\vec{w}_7)$ & $m_7$ 
\end{tabular}
\caption{Primer pravilnostne tabele in zaporedja mintermov za preklopno funkcijo treh vhodnih spremenljivk.}
\label{tab:mintermi}
\end{table}

\begin{zgled}
Zapiši minterm 9, pri 4-ih vhodnih spremenljivkah:
\end{zgled}

\begin{resitev}
Velja torej:
\begin{itemize}
\item $n=4$,
\item $i = 9_{[10]} = 1001_{[2]}$,
\item $m_9 = x_1^1 x_2^0 x_3^0 x_4^1 = x_1 \ol x_2 \ol x_3 x_4$.
\end{itemize}
\end{resitev}

\subsection{Popolna konjunktivna normalna oblika (PKNO)}
\begin{itemize}
\item Konjunktivna: operator 2. nivoja je konjunkcija ($\&$).
\item $f(x_1,x_2,...,x_n)=\&_{i=0}^{2^{n}-1}\left( M_{2^{n}-1-i} \vee f(\vec{w}_i)\right)$
\item maksterm je disjunktiven izraz - vhodne spremenljivke so povezane preko disjunkcije
\item $M_{2^{n}-1-i}$ ... maksterm $2^{n}-1-i$; $M_{2^{n}-1-i} = x_1^{\ol w_{1,i}} \vee x_2^{\ol w_{2,i}} \vee ... \vee x_n^{\ol w_{n,i}}; i=0,1,2,...,2^n-1$
\item PKNO lahko zapišemo v krajši obliki: $f(x_1,x_2,...,x_n)=\&^n(i_m,i_{m-1},...,i_1)$, kjer $i_m,i_{m-1},...,i_1$ določajo indekse makstermov, ki nastopajo v PKNO.
\end{itemize}

Recept: pri določanju PKNO konjunktivno vežemo tiste maksterme, pri katerih je funkcijska vrednost 0. Pri tem se $(2^n-1-i)$-ti maksterm nanaša na funkcijsko vrednost $f(\vec{w}_i)$ (glej tabelo \ref{tab:makstermi}).


\begin{table}[ht]
\centering
\begin{tabular}{ccc|c|c}
$x_1$ & $x_2$ & $x_3$ & f($x_1$,$x_2$,$x_3$) & maksterm\\
\hline
0 & 0 & 0 & $f(\vec{w}_0)$ & $M_7$ \\
0 & 0 & 1 & $f(\vec{w}_1)$ & $M_6$ \\
0 & 1 & 0 & $f(\vec{w}_2)$ & $M_5$ \\
0 & 1 & 1 & $f(\vec{w}_3)$ & $M_4$ \\
1 & 0 & 0 & $f(\vec{w}_4)$ & $M_3$ \\
1 & 0 & 1 & $f(\vec{w}_5)$ & $M_2$ \\
1 & 1 & 0 & $f(\vec{w}_6)$ & $M_1$ \\
1 & 1 & 1 & $f(\vec{w}_7)$ & $M_0$ \\
\end{tabular}
\caption{Primer pravilnostne tabele in zaporedja makstermov za preklopno funkcijo treh vhodnih spremenljivk.}
\label{tab:makstermi}
\end{table}

\begin{zgled}
Zapiši maksterm 9 pri 4-ih vhodnih spremenljivkah.
\end{zgled}
\begin{resitev}
Velja torej:
\begin{itemize}
\item $n=4$,
\item $i = 2^n-1 - 9=15-9=6$,
\item $6_{[10]} = 0110_{[2]}$,
\item $M_9 = x_1^{\ol 0} \vee x_2^{\ol 1} \vee x_3^{\ol 1} \vee x_4^{\ol 0} = x_1 \vee \ol x_2 \vee \ol x_3 \vee x_4$.
\end{itemize}
\end{resitev}

\begin{zgled}
Podano funkcijo pretvori v popolno disjunktivno normalno obliko na dva načina:
\begin{itemize}
\item{s pomočjo pravilnostne tabele},
\item{analitično z razširitvijo}.
\end{itemize}
\end{zgled}
\begin{resitev}
Funkcija:
$$
f(x_1,x_2,x_3) = x_1 \vee x_1 \ol x_2 \vee \ol x_2 \ol x_3
$$
je že zapisana v \textit{disjunktivni normalni obliki}, ki pa ni popolna. 

\bigskip

\textbf{Pretvorba s pomočjo pravilnostne tabele}

Zapišimo pravilnostno tabelo, s pomočjo katere lahko zapišemo PDNO, tako da prepišemo tiste minterme, pri katerih je vrednost funkcije enaka 1.\\
\begin{table}[ht]
\centering
\begin{tabular}{ccc|c|c|c}
$x_1$ & $x_2$ & $x_3$ & f($x_1$,$x_2$,$x_3$) & minterm & maksterm\\
\hline
0 & 0 & 0 & 1 & $m_0$ & $M_7$ \\
0 & 0 & 1 & 0 & $m_1$ & $M_6$ \\
0 & 1 & 0 & 0 & $m_2$ & $M_5$ \\
0 & 1 & 1 & 0 & $m_3$ & $M_4$ \\
1 & 0 & 0 & 1 & $m_4$ & $M_3$ \\
1 & 0 & 1 & 1 & $m_5$ & $M_2$ \\
1 & 1 & 0 & 1 & $m_6$ & $M_1$ \\
1 & 1 & 1 & 1 & $m_7$ & $M_0$ \\
\end{tabular}
\caption{Pravilnostna tabela funkcije $f(x_1,x_2,x_3) = x_1 \vee x_1 \ol x_2 \vee \ol x_2 \ol x_3$.}
\end{table}


Če izpišemo samo indekse teh mintermov, dobimo skrajšano obliko PDNO:\\
\hspace*{5mm} $f(x_1,x_2,x_3)=\vee^3(0,4,5,6,7)$. \\

\bigskip
Zapišimo PDNO še v eksplicitni obliki: \\
\hspace*{5mm} $f(x_1,x_2,x_3)=\ol x_1 \ol x_2 \ol x_3 \vee x_1 \ol x_2 \ol x_3 \vee x_1 \ol x_2 x_3 \vee x_1 x_2 \ol x_3 \vee x_1 x_2 x_3$. \\

\textbf{Analitična pretvorba z razširitvijo}

Funkcijo razširimo v popolno (na prvem nivoju morajo nastopati vse vhodne spremenljivke):
\begin{align*}
&x_1 \vee x_1 \ol x_2 \vee \ol x_2 \ol x_3 \\
= & x_1(x_2 \vee \ol x_2)(x_3 \vee \ol x_3) \vee x_1 \ol x_2 (x_3 \vee \ol x_3) \vee (x_1 \vee \ol x_1)\ol x_2 \ol x_3 \\
= & x_1 x_2 x_3 \vee x_1 x_2 \ol x_3 \vee x_1 \ol x_2 x_3 \vee x_1 \ol x_2 \ol x_3 \vee \hsout{x_1 \ol x_2 x_3} \vee \hsout{x_1 \ol x_2 \ol x_3} \vee \hsout{x_1 \ol x_2 \ol x_3} \vee \ol x_1 \ol x_2 \ol x_3\\
= & x_1 x_2 x_3 \vee x_1 x_2 \ol x_3 \vee x_1 \ol x_2 x_3 \vee x_1 \ol x_2 \ol x_3 \vee \ol x_1 \ol x_2 \ol x_3 \\
= & \vee^3(7,6,5,4,0) \\
= & \vee^3(0,4,5,6,7) \\
\end{align*}

\end{resitev}

\subsection{Relaciji med makstermi in mintermi}
Med mintermi in makstermi zaradi DeMorganovega pravila veljata sledeči relaciji: 
\begin{itemize}
\item $\ol m_i = M_{2^n-1-i},$ 
\item $\ol M_i = m_{2^n-1-i}.$
\end{itemize}

Relaciji lahko uporabimo pri pretvarjanju funkcije iz PDNO v PKNO in obratno. Pri podanem zapisu najprej pogledamo kateri termi manjkajo:
\begin{itemize}
\item Če je funkcija podana v PDNO, izpišemo manjkajoče minterme - v pravilnostni tabeli so soležni z enicami.
\item Če je funkcija podana v PKNO, izpišemo manjkajoče maksterme - v pravilnostni tabeli so soležni z ničlami.
\end{itemize}
S tem smo dobili pozicije termov, ki nastopajo v iskani obliki zapisa preklopne funkcije. Indekse iskanih termov dobimo tako, da nad izpisanimi mintermi oziroma makstermi uporabimo zgoraj navedeni relaciji.

\subsection{Pretvarjanje med PDNO in PKNO}
Dva možna načina reševanja:
\begin{enumerate}
\item zapišemo pravilnostno tabelo in iz nje razberemo rešitev,
\item z upoštevanjem relacij med mintermi in makstermi.
\end{enumerate}

\begin{zgled}
Funkcijo podano v PDNO pretvori v PKNO:
\begin{itemize}
\item s pomočjo pravilnostne tabele,
\item z upoštevanjem relacij med mintermi in makstermi.
\end{itemize}
\end{zgled}
\begin{resitev}
Demonstrirajmo oba načina pretvorbe.

\textbf{Pretvorba iz PDNO v PKNO: s pomočjo pravilnostne tabele} \\
Glej tabelo v prejšnjem zgledu!

Prepišemo maksterme, pri katerih je vrednost funkcije enaka 0.

\hspace*{5mm} PKNO: $\&^3(6,5,4)$

Eksplicitna PKNO: \\
\hspace*{5mm} PKNO: $(x_1 \vee x_2 \vee \ol x_3) (x_1 \vee \ol x_2 \vee x_3) (x_1 \vee \ol x_2 \vee \ol x_3)$.\\

\textbf{Pretvorba iz PDNO v PKNO: z upoštevanjem relacij med mintermi in makstermi}

Pretvorimo $\vee^3(0,4,5,6,7)$ v PKNO:
\begin{enumerate}
\item zapišemo manjkajoče minterme: $m_1,m_2,m_3$.
\item izračunamo indekse makstermov: $M_6, M_5, M_4$.
\item rešitev: $\&^3(6,5,4)$.
\end{enumerate}

Na podoben način, bi lahko pretvarjali tudi iz PKNO v PDNO.

\end{resitev}


%\section*{Laboratorijske vaje}
%S programom \textit{Logisim} realiziraj preklopno funkcijo
%$$
%f(x_1,x_2,x_3) = \&^3(6,4,1,0)
%$$
%in s simulacijo preveri pravilnost sheme.
%
%\bigskip
%Eksplicitna 
%Minimalna oblika: $f(x_1,x_2,x_3) = \ol x_1 \ol x_3 \vee x_1 \ol x_2$.\\
%Pravilnostna tabela: \\
%\begin{table}[ht]
%\centering
%\begin{tabular}{ccc|c}
%$x_1$ & $x_2$ & $x_3$ & f($x_1$,$x_2$,$x_3$) \\
%\hline
%0 & 0 & 0 & 1 \\
%0 & 0 & 1 & 0 \\
%0 & 1 & 0 & 1 \\
%0 & 1 & 1 & 0 \\
%1 & 0 & 0 & 1 \\
%1 & 0 & 1 & 1 \\
%1 & 1 & 0 & 0 \\
%1 & 1 & 1 & 0 \\
%\end{tabular}
%\end{table}
%
%Shema:\\
%\begin{figure}[ht]
%\centering
%\includegraphics[width=0.5\linewidth]{slika_v2_2.eps}
%\end{figure}
%\end{itemize}

%\subsection*{Popolna konjunktivna normalna oblika (PKNO)}
%\begin{itemize}
%\item Konjunktivna: operator 2. nivoja je konjunkcija($ \&$).
%\item $f(x_1,x_2,...,x_n)=\&_0^{2^{n}-1} \left( M_{2^{n}-1-i} \vee f(\vec{w}_i) \right)$
